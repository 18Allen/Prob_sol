\section*{Chapter7: Independece}

This is a simple demo for you to take fancy notes in \LaTeX!

\subsection{Definition}
We now see some common environment you'll need to complete your note.

\begin{answer}
	7.1.3.(AH) The 'only if' part is equivalent to definition 7.1.2, so showing the 'if' part is enough. We can start with index $J$ with size 2, then use induction to generalize the fact to any finite size $J$.
	\\
	$|J|= 2$. W.L.O.G, we set $J_2 = \{j_1,j_2\} \subset I$($J_2$ means $J$ with size 2). Since cases with one of the event being $\Omega, \emptyset$ is trivial, what is left for showing independence of the events are $A_{j_1}\cap A_{j_2}^c$ and $ A_{j_1}^c \cap A_{j_2}$. 
	\begin{equation*}
		\begin{aligned}
			\mathbb{P}(A_{j_1} \cap A_{j_2}^{c}) &=  \mathbb{P}(A_{j_1}) - \mathbb{P}(A_{j_1} \cap A_{j_2})  \qquad && \text{(Disjoint additivity)}
			\\ &= \mathbb{P}(A_{j_1}) - \mathbb{P}(A_{j_1})\mathbb{P}(A_{j_2}) && \text{(Given condition)}
			\\ &= \mathbb{P}(A_{j_1})(1 - \mathbb{P}(A_{j_2})) && \text{(Disjoint additivity)}
			\\ &= \mathbb{P}(A_{j_1})\mathbb{P}(A_{j_2}^c) 
		\end{aligned}
	\end{equation*}
	The process of $A_{j_1}^c \cap A_{j_2}$ is the same, and thus $A_{j_1}$ and $A_{j_2}\}$ are independent.  
	Now, suppose the statement is true for $J$ with size up to $n-1$, we consider $J_n = J_{n-1} \cup \{j_n\}$. Again, we skip cases with $A_{j_n},\Omega, \emptyset$,
	\begin{equation*}
		\begin{aligned}
			\mathbb{P}((\cap_{k=1}^{n-1} A_{j_k}) \cap A_{j_n}^c)  &= [\prod_{k=1}^{n-1}\mathbb{P}(A_{j_k})](1 - \mathbb{P}(A_{j_n}^c))
			\\ &=  [\prod_{k=1}^{n-1}\mathbb{P}(A_{j_k})] \mathbb{P}(A_{j_n}^c) 
		\end{aligned}
	\end{equation*}
	By mathematical induction, the given statement is true for all finite $J \subset I$. 
\end{answer}


\subsubsection{Internal Link}
You should see all the common usages of internal links. Additionally, we can use citations as \cite{newton1726philosophiae}, which just link
to the reference page!
